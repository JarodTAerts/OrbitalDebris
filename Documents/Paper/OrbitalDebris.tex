\documentclass[11pt]{article}

\usepackage[a4paper,top=1in,bottom=1in,left=1in,right=1in,marginparwidth=0.25in]{geometry}
\usepackage{graphicx}
\usepackage{wrapfig}
\usepackage{titling}
\usepackage{multicol}
\graphicspath{ {./images/} }

%Commands
\newcommand\tab[1][1cm]{\hspace*{#1}}
\newcommand\tabh[1][0.5cm]{\hspace*{#1}}

\setlength{\droptitle}{-4em}     % Eliminate the default vertical space
\addtolength{\droptitle}{-4pt}   % Only a guess. Use this for adjustment

\author{Jarod Aerts\vspace{-2ex}% Toggle commenting out the command
}
\title{Orbital Debris and its Effect on Isolation\vspace{-2ex}% to see the effect
}

\begin{document}
\maketitle

\begin{center}
	\section*{Abstract} \label{abstract}
	Since the dawn of the space age in the late 1950s there have been thousands of tons of material put into orbit. Though some has fallen back to earth, there is a growing amount of debris in orbit. This debris is growing at an ever increasing rate due to the introduction of cheap commercial launch capability. It is possible that a large amount of debris in orbit could block solar radiation and decrease the amount of energy received by the earth. This could lead to lower global temperatures. Some have even suggested purposefully putting orbital reflectors to combat global warming. Using advanced computer modeling I have mapped the orbits of over 10,000 tracked pieces of debris and calculated the solar energy they block. This provides the best view yet on how orbital debris is effecting the solar energy reaching our planet.
\end{center}

\begin{multicols*}{2}

\section{Introduction} \label{intro}
Over the past seventy years low earth orbit (LEO) has become increasingly crowded with decommissioned satellites, expelled rocket fuel, and empty upper rocket stages. All of these fall under the category of orbital debris. While a portion of the material that is put into space eventually falls back to earth, some is placed in orbits to high or stable to be drawn back into the atmosphere. To date 13,367.7 metric tons have been placed into orbit. \cite{John} This includes all functioning spacecraft and orbital debris. This great reach into space has given countless benefits to human society, but growing amounts of orbital debris breed concern over our future in space. Along side these concerns about future space travel comes the potential for debris in orbit to block incoming solar radiation.

There has been some mention in the literature of using orbit based reflectors to decrease the amount of energy the earth receives from the sun and hence curb the warming of our planet. \cite{john} There is already a significant mass and surface area of debris in orbit, and this could provide a benchmark for the effectiveness of this technique of climate control. Computer modeling has been extensively used in the study or orbital debris. Various modeling techniques have given us clear pictures of position, density, and growth of orbital debris.\cite{john} Unfortunately, there has been no research made into modeling the relationship between orbital debris and solar energy received by earth. Mapping this relationship is difficult due to the many complicated movements involved. Orbital debris in orbit will not always be blocking solar radiation, and will at different times be blocking isolation at different latitudes causing varied effects. 

In this paper I address the relationship between orbital debris and insolation. I do this through accurate and efficient modeling of the orbits of debris and the solar energy they block at each latitude. This provides a clear picture not only of how much solar energy is blocked by orbital debris, but also how much orbital debris is blocked at each latitude. It is well known that different latitudes recieve vastly different amounts of solar energy each year.\cite{john} Therefore, different latitudes would be effected different amounts due to the proportion of energy blocked to the total energy received by that latitude.

The remainder of this paper will be organized as follows. Section 2 will provide background information that will be helpful in understanding the model that was developed. Section 3 will discuss the computer model that was developed in detail. Section 4 will give results and findings of the research. Finally, section 5 provides concluding thoughts and future work.

\section{Background} \label{complexity_explaintion}
This section provides in depth explanation of orbital debris and the various orbits that are relevant for my research.

Orbital debris is essentially junk circling earth.\cite{nasa} The majority of the mass of orbital debris falls into two main categories. First, the dead or destroyed satellites. Satellites are typically launched with the expectation they will last a few years before being decommissioned. The satellites in the lowest orbits will fall back to earth after this period, but many are placed in orbits high enough there is little hope they will fall back to earth in the near future. There are also satellites that are involved in orbital collisions, destroying the satellite and leaving a cloud of debris orbiting earth. Both decommissioned and destroyed satellites can remain in orbit for hundreds or thousands of years. Second, the upper stages of the rockets which brought satellites into orbit. Launching a satellite into orbit is a complicated process, and with current technology it is impossible to get anything into orbit without multiple rocket stages, or sections. After one stage is emptied of fuel it is dropped and the other stages continue on. Finally, when the satellite is released into orbit the final rocket stage remains in orbit with it. This is because the final stage has the same speed and trajectory as the satellite itself. Recently, it has become more common for upper stages to be de-orbited, though there are still many orbits where this is not possible, and many upper rocket stages still in orbit from previous launches.

Along with these two primary components of orbital debris there is the much more numerous category of small "junk." This is composed of flecks of paint, leaked fuel, pieces of spacecraft, even junk from human spaceflight. This debris is typically less than a few centimeters in cross section, but most is only a few millimeters or less. It is this section of orbital debris that is a big unknown. It is very difficult to track such small pieces of debris and it is estimated that there could be millions of pieces of debris this small.\cite{nasa} This category of orbital debris is also important when looking at the amount of solar energy block. Though each individual piece might be small, when taken together these small pieces of debris have a large cross sectional surface area and can block a significant amount of energy. It is even been proposed to use small dust particles like this category of orbital debris in place of orbital reflectors to control earths climate. \cite{dust}



\section{Modeling Orbital Debris} \label{systems}
Lorem ipsum dolor sit amet, consectetur adipiscing elit. Nam eget congue enim, nec iaculis ipsum. Quisque pharetra in urna eu porttitor. Aenean quis lorem urna. Nullam tellus dolor, posuere sit amet pellentesque in, lacinia non sapien. Duis convallis ipsum molestie, congue turpis et, dapibus leo. Nam nec purus ut turpis tincidunt tempor gravida at lacus. Nam purus felis, maximus nec porta quis, gravida quis quam. Aliquam sit amet lacus lorem. Nam nisi urna, dapibus sed leo quis, consectetur tincidunt purus. Cras a orci id augue volutpat blandit eu vitae erat.

Curabitur et odio tempus, aliquet sem tempus, pellentesque sapien. Proin ligula massa, porta bibendum velit et, volutpat venenatis sapien. Integer augue ante, laoreet vel vehicula quis, accumsan scelerisque tortor. Duis sed scelerisque elit. Sed eu libero at est blandit sodales sit amet sit amet metus. Integer semper varius mauris volutpat viverra. In quis dui non odio auctor porttitor. Nulla facilisi. Fusce venenatis convallis sapien eget ullamcorper. Donec ultrices tincidunt commodo. Integer eget augue lorem.

\section{Results} \label{applications}
Phasellus scelerisque enim mi, a dictum leo scelerisque id. Vestibulum ante ipsum primis in faucibus orci luctus et ultrices posuere cubilia Curae; Curabitur in tempor magna, a sodales nunc. Class aptent taciti sociosqu ad litora torquent per conubia nostra, per inceptos himenaeos. Vivamus feugiat venenatis ipsum ut venenatis. Pellentesque quis scelerisque nibh. Duis in convallis tellus. Aliquam ultricies ut lacus eu sodales. Sed sit amet tristique ex. Praesent vitae porttitor augue. Nulla et purus purus.

Nulla luctus posuere leo vitae bibendum. Duis scelerisque ut elit ornare mollis. Maecenas venenatis nunc dolor, ac condimentum dolor consectetur vel. Cras euismod sem non pellentesque ultricies. Nulla viverra laoreet tempor. Praesent lacinia vulputate diam sed imperdiet. Quisque blandit ipsum in felis volutpat porta. Integer urna felis, fringilla ut congue ut, mollis id lorem. Proin mattis scelerisque augue, a cursus purus venenatis blandit.

\section{Conslusion} \label{finance}
Orci varius natoque penatibus et magnis dis parturient montes, nascetur ridiculus mus. Nunc a arcu odio. Nunc suscipit diam nisi, a aliquam elit condimentum ac. Vestibulum mollis velit vitae volutpat viverra. Vivamus faucibus arcu a metus varius, in porttitor mauris commodo. Vestibulum sit amet auctor purus. Nulla quam enim, pellentesque nec urna eu, egestas efficitur diam. Fusce ac volutpat nisl. Aenean eget lectus nibh. Pellentesque aliquam faucibus nibh eu faucibus. Aliquam ultrices at magna et gravida.

\end{multicols*}


\end{document}
